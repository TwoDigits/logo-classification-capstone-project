\documentclass{article}

\usepackage{tikz}
\usetikzlibrary{shapes,arrows,calc,matrix}
\begin{document}
\pagestyle{empty}

\def\layersep{2.5cm}

\begin{tikzpicture}[shorten >=1pt,->,draw=black!50, node distance=\layersep]
    \tikzstyle{every pin edge}=[<-,shorten <=1pt]
    \tikzstyle{neuron}=[circle,fill=black!25,minimum size=17pt,inner sep=0pt]
    \tikzstyle{input neuron}=[neuron, fill=green!50];
    \tikzstyle{output neuron}=[neuron, fill=red!50];
    \tikzstyle{hidden neuron}=[neuron, fill=blue!50];
    \tikzstyle{annot} = [text width=4em, text centered]

    % Draw the input layer nodes
    \foreach \name / \y in {1,...,4}
    % This is the same as writing \foreach \name / \y in {1/1,2/2,3/3,4/4}
        \node[input neuron, pin=left:Input \#\y] (I-\name) at (0,-\y) {};

    % Draw the first hidden layer nodes
    \foreach \name / \y in {1,...,5}
        \path[yshift=0.5cm]
            node[hidden neuron] (H1-\name) at (\layersep,-\y cm) {};
            
    % Draw the second hidden layer nodes
    \foreach \name / \y in {1,...,3}
        \path[yshift=0.5cm]
            node[hidden neuron] (H2-\name) at (2*\layersep,-\y cm) {};        

    % Draw the output layer node
    \foreach \name / \y in {1,...,2}
    	\path[yshift=0.5cm]
    		node[output neuron,pin={[pin edge={->}]right:Output \#\y}] (O-\name) at (3*\layersep,-\y) {};

    % Connect every node in the input layer with every node in the
    % first hidden layer.
    \foreach \source in {1,...,4}
        \foreach \dest in {1,...,5}
            \path (I-\source) edge (H1-\dest);
            
    % Connect every node in the first hidden layer with every node in the
    % second hidden layer.
    \foreach \source in {1,...,5}
        \foreach \dest in {1,...,3}
            \path (H1-\source) edge (H2-\dest);

    % Connect every node in the second hidden layer with the output layer
    \foreach \source in {1,...,3}
    	\foreach \dest in {1,...,2}
        	\path (H2-\source) edge (O-\dest);

    % Annotate the layers
    \node[annot,above of=H1-1, node distance=1cm] (h1) {Hidden layer};
    \node[annot,above of=H2-1, node distance=1cm] (h2) {Hidden layer};
    \node[annot,left of=h1] {Input layer};
    \node[annot,right of=h2] {Output layer};
\end{tikzpicture}
% End of code





\begin{tikzpicture}[shorten >=1pt,->,draw=black!50, node distance=\layersep]
    \tikzstyle{every pin edge}=[<-,shorten <=1pt]
    \tikzstyle{neuron}=[circle,fill=black!25,minimum size=17pt,inner sep=5pt]
    \tikzstyle{input neuron}=[neuron, fill=green!50];
    \tikzstyle{output neuron}=[neuron, fill=red!50];
    \tikzstyle{hidden neuron}=[neuron, fill=blue!50];
    \tikzstyle{annot} = [text width=4em, text centered]

    % Draw the input layer nodes
    %\foreach \name / \y in {1,...,4}
    % This is the same as writing \foreach \name / \y in {1/1,2/2,3/3,4/4}
    %   \node[input neuron, pin=left:Input \#\y] (I-\name) at (0,-\y) {};

	\begin{scope}[yshift=0cm]
        \matrix[nodes={draw,ball}]{
            \node[input neuron, pin=left:Input \#1] (I-1) {}; \\
            \node[input neuron, pin=left:Input \#2] (I-2) {}; \\
            \node[input neuron, pin=left:Input \#3] (I-3) {}; \\
            \node[input neuron, pin=left:Input \#4] (I-4) {}; \\
            };
    \end{scope}   


    % Draw the first hidden layer nodes
    %\foreach \name / \y in {1,...,5}
    %    \path[yshift=0.5cm]
    %        node[hidden neuron] (H1-\name) at (\layersep,-\y cm) {};
    \begin{scope}[yshift=0cm,xshift=2.5cm]
        \matrix[nodes={draw,ball}]{
            \node[hidden neuron] (H1-1) {}; \\
            \node[hidden neuron] (H1-2) {}; \\
            \node[hidden neuron] (H1-3) {}; \\
            \node[hidden neuron] (H1-4) {}; \\
            \node[hidden neuron] (H1-5) {}; \\
            };
    \end{scope}   
         
            
    % Draw the second hidden layer nodes
    %\foreach \name / \y in {1,...,3}
    %    \path[yshift=0.5cm]
    %        node[hidden neuron] (H2-\name) at (2*\layersep,-\y cm) {};        
	\begin{scope}[yshift=0cm,xshift=4cm]
        \matrix[nodes={draw,ball}]{
            \node[hidden neuron] (H2-1) {}; \\
            \node[hidden neuron] (H2-2) {}; \\
            \node[hidden neuron] (H2-3) {}; \\
            };
    \end{scope} 
    % Draw the output layer node
    %\foreach \name / \y in {1,...,2}
    %	\path[yshift=0.5cm]
    %		node[output neuron,pin={[pin edge={->}]right:Output \#\y}] (O-\name) at (3*\layersep,-\y) {};
	\begin{scope}[yshift=0cm,xshift=6.5cm]
        \matrix[nodes={draw,ball}]{
            \node[output neuron,pin={[pin edge={->}]right:Output \#1}] (O-1) {}; \\
            \node[output neuron,pin={[pin edge={->}]right:Output \#2}] (O-2) {}; \\
            };
    \end{scope} 
    % Connect every node in the input layer with every node in the
    % first hidden layer.
    \foreach \source in {1,...,4}
        \foreach \dest in {1,...,5}
            \path (I-\source) edge (H1-\dest);
            
    % Connect every node in the first hidden layer with every node in the
    % second hidden layer.
    \foreach \source in {1,...,5}
        \foreach \dest in {1,...,3}
            \path (H1-\source) edge (H2-\dest);

    % Connect every node in the second hidden layer with the output layer
    \foreach \source in {1,...,3}
    	\foreach \dest in {1,...,2}
        	\path (H2-\source) edge (O-\dest);

    % Annotate the layers
%    \node[annot,above of=H1-1, node distance=1cm] (h1) {Hidden layer};
%    \node[annot,above of=H2-1, node distance=1cm] (h2) {Hidden layer};
%    \node[annot,left of=h1] {Input layer};
%    \node[annot,right of=h2] {Output layer};
	\node[annot,above of=O-1, node distance=1.71cm] (o1) {Output layer};
    \node[annot,above of=H1-1, node distance=0.8cm]  (h1) {Hidden layer};
    \node[annot,above of=H2-1, node distance=1.41cm]  (h2) {Hidden layer};
    \node[annot,above of=I-1, node distance=1.1cm]  (i1) {Input layer};
\end{tikzpicture}
% End of code










\\
\\
\\
\\
\\
\begin{tikzpicture}[shorten >=1pt,->,draw=black!50, node distance=\layersep]
    \tikzstyle{every pin edge}=[<-,shorten <=1pt]
    \tikzstyle{neuron}=[circle,fill=black,minimum size=17pt,inner sep=0pt]
	\tikzstyle{act neuron}=[neuron, fill=white!50];    
    
    \tikzstyle{filter1act neuron}=[neuron, fill=white, draw=green!50, ultra thick];
    \tikzstyle{filter1deact neuron}=[neuron, draw=green!50, ultra thick];
    \tikzstyle{filter1 neuron}=[neuron, fill=green!50];
    \tikzstyle{filter2act neuron}=[neuron, fill=white, draw=blue!50, ultra thick];
    \tikzstyle{filter2deact neuron}=[neuron, draw=blue!50, ultra thick];
    \tikzstyle{filter2 neuron}=[neuron, fill=blue!50];
    \tikzstyle{filter3act neuron}=[neuron, fill=white, draw=yellow!50, ultra thick];
    \tikzstyle{filter3deact neuron}=[neuron, draw=yellow!50, ultra thick];
    \tikzstyle{filter3 neuron}=[neuron, fill=yellow!50];
    \tikzstyle{filter4act neuron}=[neuron, fill=white, draw=red!50, ultra thick];
    \tikzstyle{filter4deact neuron}=[neuron, draw=red!50, ultra thick];
    \tikzstyle{filter4 neuron}=[neuron, fill=red!50];
    
    \tikzstyle{annot} = [text width=15em, text centered]

    % Draw the first filter
    %\foreach \name / \y in {1,...,4}
    % This is the same as writing \foreach \name / \y in {1/1,2/2,3/3,4/4}
    
	\begin{scope}[yshift=1.5cm]
        \matrix[nodes={draw,ball}]{
            \node[filter1deact neuron] (F1-1) {}; &
            \node[filter1act neuron] (F1-2) {}; \\
            \node[filter1deact neuron] (F1-3) {}; &
            \node[filter1deact neuron] (F1-4) {}; \\
            };
    \end{scope}    
    \begin{scope}[yshift=1.5cm,xshift=1.5cm]
        \matrix[nodes={draw,ball}]{
            \node[filter2act neuron] (F2-1) {}; &
            \node[filter2act neuron] (F2-2) {}; \\
            \node[filter2deact neuron] (F2-3) {}; &
            \node[filter2act neuron] (F2-4) {}; \\
            };
    \end{scope}    
    
    \begin{scope}[]
        \matrix[nodes={draw,ball}]{
            \node[filter3deact neuron] (F3-1) {}; &
            \node[filter3deact neuron] (F3-2) {}; \\
            \node[filter3deact neuron] (F3-3) {} ; &
            \node[filter3act neuron] (F3-4) {}; \\
            };
    \end{scope}    
    \begin{scope}[xshift=1.5cm]
        \matrix[nodes={draw,ball}]{
            \node[filter4act neuron] (F4-1) {}; &
            \node[filter4deact neuron] (F4-2) {}; \\
            \node[filter4deact neuron] (F4-3) {}; &
            \node[filter4deact neuron] (F4-4) {}; \\
            };
    \end{scope}    
    
    
   

	\begin{scope}[yshift=-0.75cm,xshift=4cm]
        \matrix[nodes={ball}]{
            \node[filter1 neuron] (H1-1) {}; &
            \node[filter2 neuron] (H1-2) {}; \\
            \node[filter3 neuron] (H1-3) {}; &
            \node[filter4 neuron] (H1-4) {}; \\
            };
    \end{scope}  
    
    \begin{scope}[yshift=0.75cm,xshift=4cm]
        \matrix[nodes={ball}]{
            \node[filter1 neuron] (H2-1) {}; &
            \node[filter2 neuron] (H2-2) {}; \\
            \node[filter3 neuron] (H2-3) {}; &
            \node[filter4 neuron] (H2-4) {}; \\
            };
    \end{scope}  
        
    \begin{scope}[yshift=2.25cm,xshift=4cm]
        \matrix[nodes={ball}]{
            \node[filter1 neuron] (H3-1) {}; &
            \node[filter2 neuron] (H3-2) {}; \\
            \node[filter3 neuron] (H3-3) {}; &
            \node[filter4 neuron] (H3-4) {}; \\
            };
    \end{scope}  
    
    
    
    
    % Draw the output layer node
    \begin{scope}[xshift=7.5cm,yshift=0.5cm]
        \matrix[nodes={draw,ball}]{
            \node[neuron] (O-1) {}; \\
            \node[neuron] (O-2) {}; \\
            \node[neuron] (O-3) {}; \\
            \node[neuron] (O-4) {}; \\
            \node[neuron] (O-5) {}; \\
            \node[neuron] (O-6) {}; \\
            \node[act neuron, pin={[pin edge={->}]right:It is a 7!}] (O-7) {}; \\
            \node[neuron] (O-8) {}; \\
            \node[neuron] (O-9) {}; \\
            };
    \end{scope}  



    % Draw the first hidden layer nodes
    %\foreach \name / \y in {1,...,5}
    %    \path[yshift=0.5cm]
    %        node[hidden neuron] (H1-\name) at (\layersep,-\y cm) {};
            
    % Draw the second hidden layer nodes
    %\foreach \name / \y in {1,...,3}
    %    \path[yshift=0.5cm]
    %        node[hidden neuron] (H2-\name) at (2*\layersep,-\y cm) {};        

    % Draw the output layer node
    %\foreach \name / \y in {1,...,2}
   % 	\path[yshift=0.5cm]
    %		node[output neuron,pin={[pin edge={->}]right:Output \#\y}] (O-\name) at (3*\layersep,-\y) {};

    % Connect every node in the input layer with every node in the
    % first hidden layer.
    \foreach \source in {1,...,4}
            \path[color=green!50] (F1-\source) edge (H1-1);
            
     \foreach \source in {1,...,4}
            \path[color=blue!50] (F2-\source) edge (H1-2);
            
     \foreach \source in {1,...,4}
            \path[color=yellow!50] (F3-\source) edge (H1-3);
            
     \foreach \source in {1,...,4}
            \path[color=red!50] (F4-\source) edge (H1-4);
            
            
     \foreach \source in {1,...,4}
            \path[color=green!50] (F1-\source) edge (H2-1);
            
     \foreach \source in {1,...,4}
            \path[color=blue!50] (F2-\source) edge (H2-2);
            
     \foreach \source in {1,...,4}
            \path[color=yellow!50] (F3-\source) edge (H2-3);
            
     \foreach \source in {1,...,4}
            \path[color=red!50] (F4-\source) edge (H2-4);
     
     
      \foreach \source in {1,...,4}
            \path[color=green!50] (F1-\source) edge (H3-1);
            
     \foreach \source in {1,...,4}
            \path[color=blue!50] (F2-\source) edge (H3-2);
            
     \foreach \source in {1,...,4}
            \path[color=yellow!50] (F3-\source) edge (H3-3);
            
     \foreach \source in {1,...,4}
            \path[color=red!50] (F4-\source) edge (H3-4);
            
    % Connect every node in the first hidden layer with every node in the
    % second hidden layer.
    %\foreach \source in {1,...,5}
    %    \foreach \dest in {1,...,3}
    %        \path (H1-\source) edge (H2-\dest);

    % Connect every node in the second hidden layer with the output layer
    \foreach \source in {1,...,4}
    	\foreach \dest in {1,...,9}
    	{
        	\path (H1-\source) edge (O-\dest);
       		\path (H2-\source) edge (O-\dest);
       		\path (H3-\source) edge (O-\dest);
       		}
    %   	\path (H1-1) edge (O-1);
    %   	\path (H1-1) edge (O-2);
    %   	\path (H1-1) -- (O-3);
    %   	\path (H1-1) -- (O-4);
    %   	\path (H1-1) -- (O-5);
    %   	\path (H1-1) -- (O-6);
    %   	\path (H1-1) -- (O-7);
    %   	\path (H1-1) -- (O-8);
    %   	\path (H1-1) -- (O-9);

    % Annotate the layers
   
    \node[annot,above of=O-1, node distance=0.75cm] (o1) {Output layer};
    \node[annot,above of=H3-1, node distance=1.15cm]  (h1) {Convolutional layer};
    \node[annot,above of=F1-2, node distance=1.9cm]  (i1) {Input layer};
\end{tikzpicture}


\\
\\
\\
\\
\\
\begin{tikzpicture}[shorten >=1pt,->,draw=black!50, node distance=\layersep]
    \tikzstyle{every pin edge}=[<-,shorten <=1pt]
    \tikzstyle{neuron}=[circle,fill=black,minimum size=17pt,inner sep=0pt]
	\tikzstyle{act neuron}=[neuron, fill=white!50];    
    
    \tikzstyle{filter1act neuron}=[neuron, fill=white, draw=green!50, ultra thick];
    \tikzstyle{filter1deact neuron}=[neuron, draw=green!50, ultra thick];
    \tikzstyle{filter1 neuron}=[neuron, fill=green!50];
    \tikzstyle{filter2act neuron}=[neuron, fill=white, draw=blue!50, ultra thick];
    \tikzstyle{filter2deact neuron}=[neuron, draw=blue!50, ultra thick];
    \tikzstyle{filter2 neuron}=[neuron, fill=blue!50];
    \tikzstyle{filter3act neuron}=[neuron, fill=white, draw=yellow!50, ultra thick];
    \tikzstyle{filter3deact neuron}=[neuron, draw=yellow!50, ultra thick];
    \tikzstyle{filter3 neuron}=[neuron, fill=yellow!50];
    \tikzstyle{filter4act neuron}=[neuron, fill=white, draw=red!50, ultra thick];
    \tikzstyle{filter4deact neuron}=[neuron, draw=red!50, ultra thick];
    \tikzstyle{filter4 neuron}=[neuron, fill=red!50];
    
    \tikzstyle{annot} = [text width=15em, text centered]

    % Draw the first filter
    %\foreach \name / \y in {1,...,4}
    % This is the same as writing \foreach \name / \y in {1/1,2/2,3/3,4/4}
    
	\begin{scope}[yshift=1.5cm]
        \matrix[nodes={draw,ball}]{
            \node[filter1act neuron] (F1-1) {}; &
            \node[filter1deact neuron] (F1-2) {}; \\
            \node[filter1act neuron] (F1-3) {}; &
            \node[filter1deact neuron] (F1-4) {}; \\
            };
    \end{scope}    
    \begin{scope}[yshift=1.5cm,xshift=1.5cm]
        \matrix[nodes={draw,ball}]{
            \node[filter2deact neuron] (F2-1) {}; &
            \node[filter2deact neuron] (F2-2) {}; \\
            \node[filter2act neuron] (F2-3) {}; &
            \node[filter2deact neuron] (F2-4) {}; \\
            };
    \end{scope}    
    
    \begin{scope}[]
        \matrix[nodes={draw,ball}]{
            \node[filter3act neuron] (F3-1) {}; &
            \node[filter3act neuron] (F3-2) {}; \\
            \node[filter3deact neuron] (F3-3) {} ; &
            \node[filter3deact neuron] (F3-4) {}; \\
            };
    \end{scope}    
    \begin{scope}[xshift=1.5cm]
        \matrix[nodes={draw,ball}]{
            \node[filter4act neuron] (F4-1) {}; &
            \node[filter4act neuron] (F4-2) {}; \\
            \node[filter4act neuron] (F4-3) {}; &
            \node[filter4deact neuron] (F4-4) {}; \\
            };
    \end{scope}    
    
    
   

	\begin{scope}[yshift=-0.75cm,xshift=4cm]
        \matrix[nodes={ball}]{
            \node[filter1 neuron] (H1-1) {}; &
            \node[filter2 neuron] (H1-2) {}; \\
            \node[filter3 neuron] (H1-3) {}; &
            \node[filter4 neuron] (H1-4) {}; \\
            };
    \end{scope}  
    
    \begin{scope}[yshift=0.75cm,xshift=4cm]
        \matrix[nodes={ball}]{
            \node[filter1 neuron] (H2-1) {}; &
            \node[filter2 neuron] (H2-2) {}; \\
            \node[filter3 neuron] (H2-3) {}; &
            \node[filter4 neuron] (H2-4) {}; \\
            };
    \end{scope}  
        
    \begin{scope}[yshift=2.25cm,xshift=4cm]
        \matrix[nodes={ball}]{
            \node[filter1 neuron] (H3-1) {}; &
            \node[filter2 neuron] (H3-2) {}; \\
            \node[filter3 neuron] (H3-3) {}; &
            \node[filter4 neuron] (H3-4) {}; \\
            };
    \end{scope}  
    
    
    % Draw the output layer node
    \begin{scope}[xshift=7.5cm,yshift=0.5cm]
        \matrix[nodes={draw,ball}]{
            \node[neuron] (O-1) {}; \\
            \node[neuron] (O-2) {}; \\
            \node[neuron] (O-3) {}; \\
            \node[act neuron, pin={[pin edge={->}]right:It is a 4!}] (O-4) {}; \\
            \node[neuron] (O-5) {}; \\
            \node[neuron] (O-6) {}; \\
            \node[neuron] (O-7) {}; \\
            \node[neuron] (O-8) {}; \\
            \node[neuron] (O-9) {}; \\
            };
    \end{scope} 

    % Draw the first hidden layer nodes
    %\foreach \name / \y in {1,...,5}
    %    \path[yshift=0.5cm]
    %        node[hidden neuron] (H1-\name) at (\layersep,-\y cm) {};
            
    % Draw the second hidden layer nodes
    %\foreach \name / \y in {1,...,3}
    %    \path[yshift=0.5cm]
    %        node[hidden neuron] (H2-\name) at (2*\layersep,-\y cm) {};        

    % Draw the output layer node
    %\foreach \name / \y in {1,...,2}
   % 	\path[yshift=0.5cm]
    %		node[output neuron,pin={[pin edge={->}]right:Output \#\y}] (O-\name) at (3*\layersep,-\y) {};

    % Connect every node in the input layer with every node in the
    % first hidden layer.
    \foreach \source in {1,...,4}
            \path[color=green!50] (F1-\source) edge (H1-1);
            
     \foreach \source in {1,...,4}
            \path[color=blue!50] (F2-\source) edge (H1-2);
            
     \foreach \source in {1,...,4}
            \path[color=yellow!50] (F3-\source) edge (H1-3);
            
     \foreach \source in {1,...,4}
            \path[color=red!50] (F4-\source) edge (H1-4);
            
            
     \foreach \source in {1,...,4}
            \path[color=green!50] (F1-\source) edge (H2-1);
            
     \foreach \source in {1,...,4}
            \path[color=blue!50] (F2-\source) edge (H2-2);
            
     \foreach \source in {1,...,4}
            \path[color=yellow!50] (F3-\source) edge (H2-3);
            
     \foreach \source in {1,...,4}
            \path[color=red!50] (F4-\source) edge (H2-4);
     
     
      \foreach \source in {1,...,4}
            \path[color=green!50] (F1-\source) edge (H3-1);
            
     \foreach \source in {1,...,4}
            \path[color=blue!50] (F2-\source) edge (H3-2);
            
     \foreach \source in {1,...,4}
            \path[color=yellow!50] (F3-\source) edge (H3-3);
            
     \foreach \source in {1,...,4}
            \path[color=red!50] (F4-\source) edge (H3-4);
            
    % Connect every node in the first hidden layer with every node in the
    % second hidden layer.
    %\foreach \source in {1,...,5}
    %    \foreach \dest in {1,...,3}
    %        \path (H1-\source) edge (H2-\dest);

    % Connect every node in the second hidden layer with the output layer
    \foreach \source in {1,...,4}
    	\foreach \dest in {1,...,9}
    	{
        	\path (H1-\source) edge (O-\dest);
       		\path (H2-\source) edge (O-\dest);
       		\path (H3-\source) edge (O-\dest);
       		}
    %   	\path (H1-1) edge (O-1);
    %   	\path (H1-1) edge (O-2);
    %   	\path (H1-1) -- (O-3);
    %   	\path (H1-1) -- (O-4);
    %   	\path (H1-1) -- (O-5);
    %   	\path (H1-1) -- (O-6);
    %   	\path (H1-1) -- (O-7);
    %   	\path (H1-1) -- (O-8);
    %   	\path (H1-1) -- (O-9);

    % Annotate the layers
   
    \node[annot,above of=O-1, node distance=0.75cm] (o1) {Output layer};
    \node[annot,above of=H3-1, node distance=1.15cm]  (h1) {Convolutional layer};
    \node[annot,above of=F1-2, node distance=1.9cm]  (i1) {Input layer};
\end{tikzpicture}





\begin{tikzpicture}[shorten >=1pt,->,draw=black!50, node distance=\layersep]
    \tikzstyle{every pin edge}=[<-,shorten <=1pt]
    \tikzstyle{neuron}=[circle,fill=black,minimum size=17pt,inner sep=0pt]
	\tikzstyle{act neuron}=[neuron, fill=white!50];    
    
    \tikzstyle{filter1act neuron}=[neuron, fill=white, draw=green!50, ultra thick];
    \tikzstyle{filter1deact neuron}=[neuron, draw=green!50, ultra thick];
    \tikzstyle{filter1 neuron}=[neuron, fill=green!50];
    \tikzstyle{filter2act neuron}=[neuron, fill=white, draw=blue!50, ultra thick];
    \tikzstyle{filter2deact neuron}=[neuron, draw=blue!50, ultra thick];
    \tikzstyle{filter2 neuron}=[neuron, fill=blue!50];
    \tikzstyle{filter3act neuron}=[neuron, fill=white, draw=red!50, ultra thick];
    \tikzstyle{filter3deact neuron}=[neuron, draw=red!50, ultra thick];
    \tikzstyle{filter3 neuron}=[neuron, fill=red!50];
    \tikzstyle{filter4act neuron}=[neuron, fill=white, draw=yellow!50, ultra thick];
    \tikzstyle{filter4deact neuron}=[neuron, draw=yellow!50, ultra thick];
    \tikzstyle{filter4 neuron}=[neuron, fill=yellow!50];
    
    \tikzstyle{annot} = [text width=15em, text centered]

    % Draw the first filter
    %\foreach \name / \y in {1,...,4}
    % This is the same as writing \foreach \name / \y in {1/1,2/2,3/3,4/4}
    
	\begin{scope}[yshift=1.4cm]
        \matrix[nodes={draw,ball}]{
            \node[filter1act neuron] (F1-1) {0.1}; &
            \node[filter1act neuron] (F1-2) {0.7}; \\
            \node[filter1act neuron] (F1-3) {-0.3}; &
            \node[filter1act neuron] (F1-4) {-0.1}; \\
            };
    \end{scope}    
    \begin{scope}[yshift=1.4cm,xshift=1.4cm]
        \matrix[nodes={draw,ball}]{
            \node[filter2act neuron] (F2-1) {-0.3}; &
            \node[filter2act neuron] (F2-2) {-0.2}; \\
            \node[filter2act neuron] (F2-3) {-0.1}; &
            \node[filter2act neuron] (F2-4) {-0.3}; \\
            };
    \end{scope}    
    
    \begin{scope}[]
        \matrix[nodes={draw,ball}]{
            \node[filter3act neuron] (F3-1) {-0.3}; &
            \node[filter3act neuron] (F3-2) {0.4}; \\
            \node[filter3act neuron] (F3-3) {0.2}; &
            \node[filter3act neuron] (F3-4) {-0.9}; \\
            };
    \end{scope}    
    \begin{scope}[xshift=1.4cm]
        \matrix[nodes={draw,ball}]{
            \node[filter4act neuron] (F4-1) {-0.5}; &
            \node[filter4act neuron] (F4-2) {-0.6}; \\
            \node[filter4act neuron] (F4-3) {0}; &
            \node[filter4act neuron] (F4-4) {-0.2}; \\
            };
    \end{scope}    
    
    \begin{scope}[yshift=4.4cm]
        \matrix[nodes={draw,ball}]{
            \node[filter1act neuron] (F1-5) {-0.7}; &
            \node[filter1act neuron] (F1-6) {0.6}; \\
            \node[filter1act neuron] (F1-7) {0.2}; &
            \node[filter1act neuron] (F1-8) {-0.5}; \\
            };
    \end{scope}    
    \begin{scope}[yshift=4.4cm,xshift=1.4cm]
        \matrix[nodes={draw,ball}]{
            \node[filter2act neuron] (F2-5) {0.1}; &
            \node[filter2act neuron] (F2-6) {0}; \\
            \node[filter2act neuron] (F2-7) {0.5}; &
            \node[filter2act neuron] (F2-8) {-0.2}; \\
            };
    \end{scope}    
    
    \begin{scope}[yshift=3cm]
        \matrix[nodes={draw,ball}]{
            \node[filter3act neuron] (F3-5) {0}; &
            \node[filter3act neuron] (F3-6) {-0.3}; \\
            \node[filter3act neuron] (F3-7) {0.8}; &
            \node[filter3act neuron] (F3-8) {0.9}; \\
            };
    \end{scope}    
    \begin{scope}[yshift=3cm,xshift=1.4cm]
        \matrix[nodes={draw,ball}]{
            \node[filter4act neuron] (F4-5) {-0.1}; &
            \node[filter4act neuron] (F4-6) {0.3}; \\
            \node[filter4act neuron] (F4-7) {0.4}; &
            \node[filter4act neuron] (F4-8) {-0.3}; \\
            };
    \end{scope}    
    

	\begin{scope}[yshift=3.7cm,xshift=4cm]
        \matrix[nodes={ball}]{
            \node[filter1act neuron] (H1-1) {0.6}; &
            \node[filter2act neuron] (H1-2) {0.5}; \\
            \node[filter3act neuron] (H1-3) {0.9}; &
            \node[filter4act neuron] (H1-4) {0.4}; \\
            };
    \end{scope}  
   \begin{scope}[yshift=0.7cm,xshift=4cm]
        \matrix[nodes={ball}]{
            \node[filter1act neuron] (H2-1) {0.7}; &
            \node[filter2act neuron] (H2-2) {-0.1}; \\
            \node[filter3act neuron] (H2-3) {0.4}; &
            \node[filter4act neuron] (H2-4) {0}; \\
            };
    \end{scope}  
   
    
    
%    % Draw the output layer node
%    \begin{scope}[xshift=7.5cm,yshift=0.5cm]
%        \matrix[nodes={draw,ball}]{
%            \node[neuron] (O-1) {}; \\
%            \node[neuron] (O-2) {}; \\
%            \node[neuron] (O-3) {}; \\
%            \node[neuron] (O-4) {}; \\
%            \node[neuron] (O-5) {}; \\
%            \node[neuron] (O-6) {}; \\
%            \node[act neuron, pin={[pin edge={->}]right:It is a 7!}] (O-7) {}; \\
%            \node[neuron] (O-8) {}; \\
%            \node[neuron] (O-9) {}; \\
%            };
%    \end{scope}  


    % Draw the first hidden layer nodes
    %\foreach \name / \y in {1,...,5}
    %    \path[yshift=0.5cm]
    %        node[hidden neuron] (H1-\name) at (\layersep,-\y cm) {};
            
    % Draw the second hidden layer nodes
    %\foreach \name / \y in {1,...,3}
    %    \path[yshift=0.5cm]
    %        node[hidden neuron] (H2-\name) at (2*\layersep,-\y cm) {};        

    % Draw the output layer node
    %\foreach \name / \y in {1,...,2}
   % 	\path[yshift=0.5cm]
    %		node[output neuron,pin={[pin edge={->}]right:Output \#\y}] (O-\name) at (3*\layersep,-\y) {};

    % Connect every node in the input layer with every node in the
    % first hidden layer.
%    \foreach \source in {1,...,4}
%            \path[color=green!50] (F1-\source) edge (H1-1);
%            
%     \foreach \source in {1,...,4}
%            \path[color=blue!50] (F2-\source) edge (H1-2);
%            
%     \foreach \source in {1,...,4}
%            \path[color=red!50] (F3-\source) edge (H1-3);
%            
%     \foreach \source in {1,...,4}
%            \path[color=yellow!50] (F4-\source) edge (H1-4);
%            
%            
%     \foreach \source in {5,...,9}
%            \path[color=green!50] (F1-\source) edge (H2-1);
%            
%     \foreach \source in {1,...,4}
%            \path[color=blue!50] (F2-\source) edge (H2-2);
%            
%     \foreach \source in {1,...,4}
%            \path[color=red!50] (F3-\source) edge (H2-3);
%            
%     \foreach \source in {1,...,4}
%            \path[color=yellow!50] (F4-\source) edge (H2-4);
%            
    % Connect every node in the first hidden layer with every node in the
    % second hidden layer.
    %\foreach \source in {1,...,5}
    %    \foreach \dest in {1,...,3}
    %        \path (H1-\source) edge (H2-\dest);

    % Connect every node in the second hidden layer with the output layer
%    \foreach \source in {1,...,4}
%    	\foreach \dest in {1,...,9}
%    	{
%        	\path (H1-\source) edge (O-\dest);
%       		\path (H2-\source) edge (O-\dest);
%       		}
%       	\path (F4-2) edge (H2-1);
%       	\path (F4-6) edge (H1-1);
\draw[->] (2.5,0.7) --(3,0.7)
\draw[->] (2.5,3.77) --(3,3.77)
    %   	\path (H1-1) -- (O-3);
    %   	\path (H1-1) -- (O-4);
    %   	\path (H1-1) -- (O-5);
    %   	\path (H1-1) -- (O-6);
    %   	\path (H1-1) -- (O-7);
    %   	\path (H1-1) -- (O-8);
    %   	\path (H1-1) -- (O-9);

    % Annotate the layers
   
%    \node[annot,above of=O-1, node distance=0.75cm] (o1) {Output layer};
%    \node[annot,above of=H2-1, node distance=1.9cm]  (h1) {Hidden layer};
%    \node[annot,above of=H1-4, node distance=3cm] (h2) {Max pooling layer};
%    \node[annot,above of=F1-2, node distance=1.9cm]  (i1) {Input layer};
\end{tikzpicture}










\begin{tikzpicture}[shorten >=1pt,->,draw=black!50, node distance=\layersep]
    \tikzstyle{every pin edge}=[<-,shorten <=1pt]
    \tikzstyle{neuron}=[circle,fill=black,minimum size=17pt,inner sep=0pt]
	\tikzstyle{act neuron}=[neuron, fill=white!50];    
    
    \tikzstyle{filter1act neuron}=[neuron, fill=white, draw=green!50, ultra thick];
    \tikzstyle{filter1deact neuron}=[neuron, draw=green!50, ultra thick];
    \tikzstyle{filter1 neuron}=[neuron, fill=green!50];
    \tikzstyle{filter2act neuron}=[neuron, fill=white, draw=blue!50, ultra thick];
    \tikzstyle{filter2deact neuron}=[neuron, draw=blue!50, ultra thick];
    \tikzstyle{filter2 neuron}=[neuron, fill=blue!50];
    \tikzstyle{filter3act neuron}=[neuron, fill=white, draw=red!50, ultra thick];
    \tikzstyle{filter3deact neuron}=[neuron, draw=red!50, ultra thick];
    \tikzstyle{filter3 neuron}=[neuron, fill=red!50];
    \tikzstyle{filter4act neuron}=[neuron, fill=white, draw=yellow!50, ultra thick];
    \tikzstyle{filter4deact neuron}=[neuron, draw=yellow!50, ultra thick];
    \tikzstyle{filter4 neuron}=[neuron, fill=yellow!50];
    
    \tikzstyle{annot} = [text width=15em, text centered]

    % Draw the first filter
    %\foreach \name / \y in {1,...,4}
    % This is the same as writing \foreach \name / \y in {1/1,2/2,3/3,4/4}
    
	\begin{scope}[]
        \matrix[nodes={draw,ball}]{
            \node[filter1act neuron] (F1-1) {0.8}; &
            \node[filter1act neuron] (F1-2) {0.4}; \\
            \node[filter1act neuron] (F1-3) {-0.3}; &
            \node[filter1act neuron] (F1-4) {-0.5}; \\
            };
    \end{scope}    
    \begin{scope}[yshift=1.4cm]
        \matrix[nodes={draw,ball}]{
            \node[filter2act neuron] (F2-1) {-0.1}; &
            \node[filter2act neuron] (F2-2) {-0.2}; \\
            \node[filter2act neuron] (F2-3) {-0.4}; &
            \node[filter2act neuron] (F2-4) {-0.1}; \\
            };
    \end{scope}    
    

	\begin{scope}[yshift=0cm,xshift=2cm]
        \matrix[nodes={ball}]{
            \node[filter1act neuron] (H1-1) {0.1}; \\
            };
    \end{scope}  
   \begin{scope}[yshift=1.4cm,xshift=2cm]
        \matrix[nodes={ball}]{
            \node[filter2act neuron] (H2-1) {-0.2}; \\
            };
    \end{scope}  
   
    
    
%    % Draw the output layer node
%    \begin{scope}[xshift=7.5cm,yshift=0.5cm]
%        \matrix[nodes={draw,ball}]{
%            \node[neuron] (O-1) {}; \\
%            \node[neuron] (O-2) {}; \\
%            \node[neuron] (O-3) {}; \\
%            \node[neuron] (O-4) {}; \\
%            \node[neuron] (O-5) {}; \\
%            \node[neuron] (O-6) {}; \\
%            \node[act neuron, pin={[pin edge={->}]right:It is a 7!}] (O-7) {}; \\
%            \node[neuron] (O-8) {}; \\
%            \node[neuron] (O-9) {}; \\
%            };
%    \end{scope}  


    % Draw the first hidden layer nodes
    %\foreach \name / \y in {1,...,5}
    %    \path[yshift=0.5cm]
    %        node[hidden neuron] (H1-\name) at (\layersep,-\y cm) {};
            
    % Draw the second hidden layer nodes
    %\foreach \name / \y in {1,...,3}
    %    \path[yshift=0.5cm]
    %        node[hidden neuron] (H2-\name) at (2*\layersep,-\y cm) {};        

    % Draw the output layer node
    %\foreach \name / \y in {1,...,2}
   % 	\path[yshift=0.5cm]
    %		node[output neuron,pin={[pin edge={->}]right:Output \#\y}] (O-\name) at (3*\layersep,-\y) {};

    % Connect every node in the input layer with every node in the
    % first hidden layer.
%    \foreach \source in {1,...,4}
%            \path[color=green!50] (F1-\source) edge (H1-1);
%            
%     \foreach \source in {1,...,4}
%            \path[color=blue!50] (F2-\source) edge (H1-2);
%            
%     \foreach \source in {1,...,4}
%            \path[color=red!50] (F3-\source) edge (H1-3);
%            
%     \foreach \source in {1,...,4}
%            \path[color=yellow!50] (F4-\source) edge (H1-4);
%            
%            
%     \foreach \source in {5,...,9}
%            \path[color=green!50] (F1-\source) edge (H2-1);
%            
%     \foreach \source in {1,...,4}
%            \path[color=blue!50] (F2-\source) edge (H2-2);
%            
%     \foreach \source in {1,...,4}
%            \path[color=red!50] (F3-\source) edge (H2-3);
%            
%     \foreach \source in {1,...,4}
%            \path[color=yellow!50] (F4-\source) edge (H2-4);
%            
    % Connect every node in the first hidden layer with every node in the
    % second hidden layer.
    %\foreach \source in {1,...,5}
    %    \foreach \dest in {1,...,3}
    %        \path (H1-\source) edge (H2-\dest);

    % Connect every node in the second hidden layer with the output layer
%    \foreach \source in {1,...,4}
%    	\foreach \dest in {1,...,9}
%    	{
%        	\path (H1-\source) edge (O-\dest);
%       		\path (H2-\source) edge (O-\dest);
%       		}
%       	\path (F4-2) edge (H2-1);
%       	\path (F4-6) edge (H1-1);
\draw[->] (1,0) --(1.4,0)
\draw[->] (1,1.4) --(1.4,1.4)
    %   	\path (H1-1) -- (O-3);
    %   	\path (H1-1) -- (O-4);
    %   	\path (H1-1) -- (O-5);
    %   	\path (H1-1) -- (O-6);
    %   	\path (H1-1) -- (O-7);
    %   	\path (H1-1) -- (O-8);
    %   	\path (H1-1) -- (O-9);

    % Annotate the layers
   
%    \node[annot,above of=O-1, node distance=0.75cm] (o1) {Output layer};
%    \node[annot,above of=H2-1, node distance=1.9cm]  (h1) {Hidden layer};
%    \node[annot,above of=H1-4, node distance=3cm] (h2) {Max pooling layer};
%    \node[annot,above of=F1-2, node distance=1.9cm]  (i1) {Input layer};
\end{tikzpicture}

\end{document}